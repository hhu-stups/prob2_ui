#macro( call $this )#set( $stfu = $this )#end
\documentclass{autodoc}
% Document
\begin{document}
\begin{titlepage}
	\begin{center}
	\includegraphics[width=0.50\textwidth]{html_files/ProB_Logo.png}\par\vspace{1cm}
	{\scshape\LARGE Automated Documentation \par}
	\vspace{1cm}
	{\scshape\Large of Project \par}
	\vspace{1.5cm}
	{\huge\bfseries $documentationUtility.latexSafe($project.getName()) \par}
	\vspace{2cm}
	\vfill

	% Bottom of the page
	{\large \today\par}
\end{center}
\end{titlepage}

\section{Introduction}
This is an automatically created documentation of the ProB2-UI project: $documentationUtility.latexSafe($project.getName()).
The $documentationUtility.latexSafe($project.getName()) project consists of  $project.getMachines().size() machines, $machines.size() of which were chosen for documentation.\\
#if(!$project.getDescription().isEmpty()) The Project Description is: $documentationUtility.latexSafe($project.getDescription()) #end

%LOOP OVER ALL MACHINES
#foreach($machine in $machines)
	% MACHINE CODE
	%Unicodes in MCH Code can be added in autodoc.cls as Literate of MCH Style
\section{$documentationUtility.latexSafe($machine.getName()) }
The B Machine $documentationUtility.latexSafe($machine.getName()) is defined by its Code, shown in Listing \ref{lst:code$velocityCount}
\subsection{Code of B Machine $documentationUtility.latexSafe($machine.getName()) }
	\begin{lstlisting}[style=MCH, caption = $documentationUtility.latexSafe($machine.getName()) MCH Code, label = lst:code$velocityCount]
$documenter.getMachineCode($machine)
	\end{lstlisting}
%MODELCHECKING
#if($modelchecking && !$machine.getModelcheckingItems().isEmpty())
\subsection{Model Checking}
\begin{flushleft}
The B Machine $documentationUtility.latexSafe($machine.getName()) has $documenter.getNumberSelectedTasks($machine.getModelcheckingItems())  Modelchecking Tasks of which $documenter.getNumberSuccessfulTasks($machine.getModelcheckingItems()) were checked successfully
$documenter.getNumberFailedTasks($machine.getModelcheckingItems()) failed and $documenter.getNumberNotCheckedTasks($machine.getModelcheckingItems()) were not checked at all.
\end{flushleft}
\tablestyle[sansbold]
	\begin{longtable}{*{2}{p{0.45\textwidth}}}
	\theadstart
		\thead Modelchecking Item &
		\thead Modelchecking Result\\
	\endfirsthead
	\tsubheadstart
		\thead Modelchecking Item &
		\thead Modelchecking Result\\
	\endhead
		\rowcolor{white}\caption{Modelchecking Tasks and Results}\\
	\endlastfoot
	\tbody
#foreach($item in $machine.getModelcheckingItems())
#if($item.selected())
#if($item.getItems().isEmpty())
	$documentationUtility.toUIString($item,$i18n) & Modelchecking not solved\\
#else
	$documentationUtility.toUIString($item,$i18n) & #foreach($result in $item.getItems()) $result.getMessage() #end \\
#end
#end
#end
	\tend
	\end{longtable}
#end
	%LTL
	#if($ltl)
	#if(!$machine.getLTLFormulas().isEmpty() || !$machine.getLTLPatterns().isEmpty() )\subsection{LTL Model Checking} #end
	#if(!$machine.getLTLFormulas().isEmpty())
	\begin{flushleft}
	The B Machine $documentationUtility.latexSafe($machine.getName()) has $documenter.getNumberSelectedTasks($machine.getLTLFormulas()) LTL Formulas of which $documenter.getNumberSuccessfulTasks($machine.getLTLFormulas()) were checked successfully
	$documenter.getNumberFailedTasks($machine.getLTLFormulas()) failed and $documenter.getNumberNotCheckedTasks($machine.getLTLFormulas()) were not checked at all.
	\end{flushleft}
\tablestyle[sansbold]
		\begin{longtable}{*{2}{p{0.45\textwidth}}}
		\theadstart
			\thead LTL Formular &
			\thead Status\\
		\endfirsthead
		\tsubheadstart
			\thead LTL Formular &
			\thead Status\\
		\endhead
			\rowcolor{white}\caption{LTL Formulars and Results}\\
		\endlastfoot
		\tbody
	#foreach($formula in $machine.getLTLFormulas())
	#if($formula.selected())
	#if($documenter.formulaHasResult($formula))
\begin{tabularlstlisting}
$formula.getCode()
\end{tabularlstlisting} &  $i18n.translate($formula.getResultItem().getHeaderBundleKey())\\
#else
\begin{tabularlstlisting}
$formula.getCode()
\end{tabularlstlisting} & Formula not solved \\
	#end
	#end
	#end
		\tend
		\end{longtable}
		\justifying
			#end
			#if(!$machine.getLTLPatterns().isEmpty())
		\tablestyle[sansbold]
		\begingroup
		\setlength{\LTleft}{-\textwidth plus -1fill}
		\setlength{\LTright}{\LTleft}
		\begin{longtable}{p{0.3\textwidth}p{0.3\textwidth}p{0.3\textwidth}}
		\theadstart
			\thead Pattern Name &
			\thead Code &
			\thead Result\\
		\endfirsthead
		\tsubheadstart
			\thead Pattern Name &
			\thead Code &
			\thead Result\\
		\endhead
			\rowcolor{white}\caption{LTL Patterns and Results}\\
		\endlastfoot
		\tbody
	#foreach($pattern in $machine.getLTLPatterns())
	#if($documenter.patternHasResult($pattern))
		$pattern.getName() &
\begin{tabularlstlisting}
$pattern.getCode()
\end{tabularlstlisting} & $i18n.translate($pattern.getResultItem().getHeaderBundleKey()) \\
	#else
		$pattern.getName() &
\begin{tabularlstlisting}
$pattern.getCode()
\end{tabularlstlisting} & Pattern not solved \\
	#end
	#end
		\tend
		\end{longtable}
		\justifying
	#end
	#end


	%SYMBOLIC
	#if($symbolic && !$machine.getSymbolicCheckingFormulas().isEmpty())
\subsection{Symbolic Model Checking}
\begin{flushleft}
The B Machine $documentationUtility.latexSafe($machine.getName()) has $documenter.getNumberSelectedTasks($machine.getSymbolicCheckingFormulas()) Symbolic Model Checking Formulas of which $documenter.getNumberSuccessfulTasks($machine.getSymbolicCheckingFormulas()) were checked successfully
$documenter.getNumberFailedTasks($machine.getSymbolicCheckingFormulas()) failed and $documenter.getNumberNotCheckedTasks($machine.getSymbolicCheckingFormulas()) were not checked at all.
\end{flushleft}
\tablestyle[sansbold]
	\begingroup
	\setlength{\LTleft}{-\textwidth plus -1fill}
	\setlength{\LTright}{\LTleft}
	\begin{longtable}{p{0.45\textwidth}p{0.3\textwidth}p{0.3\textwidth}}
	\theadstart
		\thead Symbolic Type&
		\thead Configuration &
		\thead Result\\
	\endfirsthead
	\tsubheadstart
		\thead Symbolic Type&
		\thead Configuration &
		\thead Result\\
	\endhead
		\rowcolor{white}\caption{Symbolic Formulars and Results}\\
	\endlastfoot
	\tbody
#foreach($sitem in $machine.getSymbolicCheckingFormulas())
#if($sitem.selected())
#if($documenter.symbolicHasResult($sitem))
	$documentationUtility.latexSafe($sitem.getType().toString()) & $documentationUtility.latexSafe($sitem.getCode()) &  $i18n.translate($sitem.getResultItem().getHeaderBundleKey()) \\
#else
	$documentationUtility.latexSafe($sitem.getType().toString()) & $documentationUtility.latexSafe($sitem.getCode())  & Formula not solved \\
#end
#end
#end
	\tend
	\end{longtable}
	\endgroup
	\justifying
	#end
#if(!$machine.getTraces().isEmpty())
\subsection{Traces}
\begin{flushleft}
A Trace refers to a sequence of state transitions, operations, and events that occur during the execution of a system.
The trace can be visualized as a table that lists the operations chronologically, with the position column indicating the index of the order, and the transition
column indicating the name of the operation. This can be useful for understanding the behavior of the system and identifying any errors or issues that may be present.
Traces are orders of operations. The B Machine $documentationUtility.latexSafe($machine.getName()) has $machine.getTraces().size() Traces.
In addition to the table, a visualization of every trace is generated via VisB. The visualization provides an interactive and visual
representation of the system's behavior during the trace transitions. #if($printHtmlCode) The HTML code for these visualization files is also included in the appendix. #end
\end{flushleft}
#foreach($trace in $machine.getTraces())
		#call($trace.load())
	\tablestyle[sansbold]
	\begin{longtable}{ll}
	\rowcolor{white}\caption{$documentationUtility.latexSafe($trace.getName())} \\
	\theadstart
		\thead Position &
		\thead Transition\\
	\endfirsthead
	\theadstart
		\thead Position &
		\thead Transition\\
	\endhead
	\tbody
#foreach($transition in $trace.getLoadedTrace().getTransitionList())
	$velocityCount & $documentationUtility.latexSafe($Transition.prettifyName($transition.getOperationName())) \\
#end
	\tend
	\end{longtable}
	#call( $traceHtmlPaths.put($trace.getName(), $documenter.saveTraceHtml($machine,$trace)) )
	\IfFileExists{$traceHtmlPaths.get($trace.getName())}{
		For the HTML Trace Visualization of $documentationUtility.latexSafe($trace.getName()) \href{run:$traceHtmlPaths.get($trace.getName())}{[click here!]}\\
	}{
	The HTML file for the Trace Visualization isn't available.
	}
	#if($printHtmlCode) For the HTML Source Code \hyperref[lst:$trace.getName()]{[click here!]}  \\ #end
	\justifying
#end
#end
#end

#if(!$traceHtmlPaths.isEmpty() && $printHtmlCode)
\appendix
\justifying
	\section{Traces Visualization HTML Code}
The visualizations are in the form of interactive HTML files. If the HTML Files are not available you can manually copy the following HTML source codesA .
	#foreach($pathEntry in $traceHtmlPaths.entrySet())
\begin{lstlisting}[label = lst:$pathEntry.key ,style = htmlAppendix, caption = $documentationUtility.latexSafe($pathEntry.key) MCH Code]
$documenter.getTraceHtmlCode($pathEntry.value)
\end{lstlisting}
	#end
#end
\end{document}
