#macro( call $this )#set( $stfu = $this )#end
\documentclass[11pt]{article}
\usepackage{color}
\usepackage{listings}
\usepackage{tabularx}
\usepackage{longtable}
\usepackage[table]{xcolor}
\usepackage{colortbl}
\usepackage{tablestyles}
\usepackage{float}
\usepackage[style=1]{mdframed}
\usepackage[utf8]{inputenc}
\usepackage[T1]{fontenc}
\usepackage{changepage}
\usepackage{ragged2e, lipsum}
\usepackage{caption, booktabs}
\usepackage{graphicx}

\lstset{literate=
		{∈}{$\in$}1
		{⇒}{$\Rightarrow$}1
		{≠}{$\neq$}1
		{∀}{$\forall$}1
		{‥}{$\cdot\cdot$}1
}

\lstset{ %
	language=C++,                % choose the language of the code
	basicstyle=\footnotesize,       % the size of the fonts that are used for the code
	numbers=left,                   % where to put the line-numbers
	numberstyle=\footnotesize,      % the size of the fonts that are used for the line-numbers
	stepnumber=1,                   % the step between two line-numbers. If it is 1 each line will be numbered
	numbersep=5pt,                  % how far the line-numbers are from the code
	backgroundcolor=\color{white},  % choose the background color. You must add \usepackage{color}
	showspaces=false,               % show spaces adding particular underscores
	showstringspaces=false,         % underline spaces within strings
	showtabs=false,                 % show tabs within strings adding particular underscores
	frame=single,           % adds a frame around the code
	tabsize=2,          % sets default tabsize to 2 spaces
	captionpos=b,           % sets the caption-position to bottom
	breaklines=true,        % sets automatic line breaking
	breakatwhitespace=false,    % sets if automatic breaks should only happen at whitespace
	escapeinside={\%*}{*)}          % if you want to add a comment within your code
}
\lstnewenvironment{tabularlstlisting}[1][]
{%
	\lstset{aboveskip=-2ex,belowskip=-3ex,#1}%
}
{}

\newcommand\cropped[1]{%
\immediate\write18{convert -trim #1 #1}%
\includegraphics[width=\textwidth]{#1}}

% Packages
% Document
\begin{document}
\begin{titlepage}
	\centering
	\includegraphics[width=0.50\textwidth]{$documenter.saveProBLogo()}\par\vspace{1cm}
	{\scshape\LARGE Automated Documentation \par}
	\vspace{1cm}
	{\scshape\Large of Project \par}
	\vspace{1.5cm}
	{\huge\bfseries $DocumentUtility.latexSafe($project.getName()) \par}
	\vspace{2cm}
	{\Large\itshape #if(!$project.getDescription().isEmpty()) \textsc{Abstract:} $project.getDescription() #end \par}
	\vfill

	% Bottom of the page
	{\large \today\par}
\end{titlepage}
#foreach($machine in $machines)
	% MACHINE CODE
	\section{$DocumentUtility.latexSafe($machine.getName()) }
	\begin{lstlisting}[caption = $DocumentUtility.latexSafe($machine.getName()) MCH Code]
	$documenter.getMachineCode($machine)
	\end{lstlisting}
	%MODELCHECKING
	#if($modelchecking && !$machine.getModelcheckingItems().isEmpty())
		\subsection{Model Checking}
		\tablestyle[sansbold]
		\begin{longtable}{*{2}{p{0.45\textwidth}}}
		\theadstart
			\thead Modelchecking Item &
			\thead Modelchecking Result\\
		\endfirsthead
		\tsubheadstart
			\thead Modelchecking Item &
			\thead Modelchecking Result\\
		\endhead
		\rowcolor{white}\caption{Modelchecking Items and Results}\\
		\endlastfoot
		\tbody
		#foreach($item in $machine.getModelcheckingItems())
			#if($item.getItems().isEmpty())
				$DocumentUtility.toUIString($item,$i18n) & Modelchecking not solved\\
			#else
				$DocumentUtility.toUIString($item,$i18n) & #foreach($result in $item.getItems()) $result.getMessage() #end \\
			#end
		#end
		\tend
		\end{longtable}
	#end
	%LTL
	#if($ltl)
		\subsection{LTL Model Checking}
		#if(!$machine.getLTLFormulas().isEmpty())
			\tablestyle[sansbold]
			\begin{longtable}{*{2}{p{0.45\textwidth}}}
			\theadstart
				\thead LTL Formular &
				\thead Status\\
			\endfirsthead
			\tsubheadstart
				\thead LTL Formular &
				\thead Status\\
			\endhead
			\rowcolor{white}\caption{LTL Formulars and Results}\\
			\endlastfoot
			\tbody
			#foreach($formula in $machine.getLTLFormulas())
				#if($formula.selected())
					#if($documenter.formulaHasResult($formula))
\begin{tabularlstlisting}
$formula.getCode()
\end{tabularlstlisting} &  $i18n.translate($formula.getResultItem().getHeaderBundleKey())\\
					#else
\begin{tabularlstlisting}
$formula.getCode()
\end{tabularlstlisting} & Formula not solved \\
					#end
				#end
			#end
		#end
		\tend
		\end{longtable}
	#end
	%SYMBOLIC
	#if($symbolic && !$machine.getSymbolicCheckingFormulas().isEmpty())
		\subsection{Symbolic Model Checking}
		\tablestyle[sansbold]
		\begingroup
		\setlength{\LTleft}{-\textwidth plus -1fill}
		\setlength{\LTright}{\LTleft}
		\begin{longtable}{p{0.45\textwidth}p{0.3\textwidth}p{0.3\textwidth}}
		\theadstart
			\thead Symbolic Type&
			\thead Configuration &
			\thead Result\\
		\endfirsthead
		\tsubheadstart
			\thead Symbolic Type&
			\thead Configuration &
			\thead Result\\
		\endhead
		\rowcolor{white}\caption{Symbolic Formulars and Results}\\
		\endlastfoot
		\tbody
		#foreach($sitem in $machine.getSymbolicCheckingFormulas())
			#if($sitem.selected())
				#if($documenter.symbolicHasResult($sitem))
					$DocumentUtility.latexSafe($sitem.getType().toString()) & $DocumentUtility.latexSafe($sitem.getCode()) &  $i18n.translate($sitem.getResultItem().getHeaderBundleKey()) \\
				#else
					$DocumentUtility.latexSafe($sitem.getType().toString()) & $DocumentUtility.latexSafe($sitem.getCode())  & Formula not solved \\
				#end
			#end
		#end
		\tend
		\end{longtable}
		\endgroup
	#end
	\subsection{Traces}
	#foreach($trace in $machine.getTraces())
		#call($trace.load())
		\tablestyle[sansbold]
		\begin{longtable}{ll}
		\rowcolor{white}\caption{$DocumentUtility.latexSafe($trace.getName())} \\
		\theadstart
			\thead Position &
			\thead Transition\\
		\endfirsthead
		\theadstart
			\thead Position &
			\thead Transition\\
		\endhead
		\tbody
		#foreach($transition in $trace.getLoadedTrace().getTransitionList())
			$velocityCount & $DocumentUtility.latexSafe($Transition.prettifyName($transition.getOperationName())) \\
		#end
		\tend
		\end{longtable}
		#foreach($image in $documenter.saveTraceImage($machine,$trace))
			\cropped{$image}
		#end
	#end
#end
\end{document}
